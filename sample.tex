%
% 公立はこだて未来大学卒業研究中間報告書[全コース対応版]
%
%         ファイル名:"sample.tex"
%
\documentclass[11pt]{bxjsarticle}
\usepackage{funinfosys}
\usepackage{url}
\usepackage[dvipdfmx]{graphicx}
\author{% 
b1020036 中川匠海\\指導教員 : 函館一郎
}
\course{Information Systems Course}

\title{キャンパスDX に向けた学務情報のオープンデータ化}
\etitle{How to Write Manuscripts for Midterm Report}
\eauthor{Taro MIRAI}
\abstract{和文は300から400文字で記述すること.和文は300から400文字で記述すること.和文は300から400文字で記述すること.和文は300から400文字で記述すること.和文は300から400文字で記述すること.和文は300から400文字で記述すること.和文は300から400文字で記述すること.和文は300から400文字で記述すること.和文は300から400文字で記述すること.和文は300から400文字で記述すること.和文は300から400文字で記述すること.和文は300から400文字で記述すること.和文は300から400文字で記述すること.和文は300から400文字で記述すること.和文は300から400文字で記述すること.和文は300から400文字で記述すること.和文は300から400文字で記述すること.和文は300から400文字で記述すること.和文は300から400文字で記述すること.}
\keywords{北海道, 函館, 亀田中野, 公立はこだて未来大学}
\eabstract{English should be written in 100 to 150 words.English should be written in 100 to 150 words.English should be written in 100 to 150 words.English should be written in 100 to 150 words.English should be written in 100 to 150 words.English should be written in 100 to 150 words.English should be written in 100 to 150 words.English should be written in 100 to 150 words.English should be written in 100 to 150 words.English should be written in 100 to 150 words.English should be written in 100 to 150 words.English should be written in 100 to 150 words.English should be written in 100 to 150 words.English should be written in 100 to 150 words.English should be written in 100 to 150 words.English should be written in 100 to 150 words.English should be written in 100 to 150 words.English should be written in 100 to 150 words.}
\ekeywords{Hokkaido, Hakodate, Kamedanakano, FUN}
\begin{document}
\maketitle
%\vspace*{-.5cm}

\section{背景と目的}

aaaaaこのサンプルは情報システムコースにおける中間報告書の様式について説明したものである.必ずしもこの雛形を使う必要はないが,仕上がりイメージはできる限りこの雛形にあわせること.

用紙サイズはA4,向きは縦とし,上下の余白は30mm、左右の余白は25mmとする.本文には明朝体とTimes New Romanを用いる.ただし,タイトルや章節の見出し,図表のキャプションはゴシック体とする.タイトルは14ポイント,氏名と章の見出しは12ポイント,節の見出しは11ポイント,その他は10ポイントとする.また,和文タイトルから英文キーワードまでは1段,本文は2段で構成とし,1段のセクションは42文字×45行,2段のセクションは20文字×45行とする.

なお,章立てはあくまでも参考であり,これに限らない.

\section{◯◯コースにおける本研究の位置づけ}
中間報告書中のいずれかの場所に,学生所属コースのカリキュラム・ポリシーに基づき,本研究の位置づけを述べる.

未来大学のカリキュラム・ポリシー
\url{https://www.fun.ac.jp/curriculum-policy} のうち,学生所属コースの項に書かれている卒業研究に関する記述を参照.

\section{関連研究}

中間報告書の文量は4ページとする.学籍番号をファイル名としたPDFファイル1つにまとめた形で作成すること.提出するファイル名はb10xxxxx.pdfとする.

句読点は「,」,「.」とする.「、」,「。」は使用しない.アブストラクトなど英文表記の部分については,スペルチェックプログラムによるチェックをする.

\section{提案する理論}

\subsection{数式}

数式による記述が必要な場合は,式番号を適切に参照しながらまとめること.

\subsection{図・写真}

読者の理解を助けるため,図や表を効果的に利用すること.図のキャプションは

\begin{center}図1 ○○○○\end{center}

のように,図の下に記す.表のキャプションは

\begin{center}表1 ○○○○\end{center}

のように,表の上に記す.

\section{実験と評価}

\section{考察}

\section{結言}

\begin{thebibliography}{99}
\bibitem{marumaru}
	○○△△, システム情報科学会論文誌, 2, 13-19, 2002.
\bibitem{abc}
	A.B.Cdddddd, J. Systems Information Science, 11, 1145-1159, 2001.
\bibitem{batubatu}
	○○××, □□△△, システム情報科学, ☆☆出版, 1999, 20-21.
\bibitem{efghij}
	E.Fggg and H.Ijjj, Electrical Engineering, KKPress, 2003, 281-284.
\end{thebibliography}
\end{document}
%
%
% EOF 
