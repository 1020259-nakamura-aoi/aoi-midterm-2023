%
% 公立はこだて未来大学卒業研究中間報告書[全コース対応版]
%
%         ファイル名:"sample.tex"
%
\documentclass[11pt]{ujarticle}
\usepackage{funinfosys}
\usepackage{url}
\usepackage[dvipdfmx]{graphicx}

\author{% 
b1020036 中川匠海\\指導教員 : 松原克弥
}
\course{Information Systems Course}

\title{キャンパスDX に向けた学務情報のオープンデータ化}
\etitle{Open Data Initiatives for Academic Affairs Information in Preparation for Campus Digital Transformation (DX)}
\eauthor{Takumi Nakagawa}
\abstract{今日の大学ではLearning Management System(LMS)や教務システムなど、目的に応じて複数のシステムを組み合わせたICT学習支援環境を構築している.しかし、複数のシステムに情報が分散していることや、Webサイトやメール等の限られたアクセス手段しか提供されないことで、学生に対する「学務情報への到達容易性」が課題となっている.
\\ 本研究では、未来大にて使用されている複数のシステムで、それぞれ異なる形式で分散管理されている学務情報(休講、補講、教室変更、振替連絡、課題〆切、教室空き情報など)をスクレイピング技術等を用いて収集し、モバイルアプリなどで活用できるようオープンデータ化し、スマートフォンやSNSなどの情報アクセス媒体になれた「デジタルネイティブ世代」の学生に適した学習支援の実現を目指す.}
\keywords{キャンパスDX,オープンデータ,公立はこだて未来大学}
\eabstract{Modern universities employ Learning Management Systems (LMS) and other systems to create comprehensive ICT learning environments. Nonetheless, the dispersion of information across various systems and limited accessibility impede students' easy retrieval of academic information.
\\In this research, conducted at Mirai University, we seek to collect disparate academic data (including class cancellations, supplementary lessons, room changes, notifications, assignment deadlines, and room availability) through scraping technology. By converting this information into open data, we aim to enhance learning support for 'digital native' students accustomed to smartphones and social networks.}
\ekeywords{CampusDX, OpenData, FUN}
\begin{document}
\maketitle
%\vspace*{-.5cm}

\section{背景と目的}



\section{◯◯コースにおける本研究の位置づけ}


\section{関連研究}


\section{提案する理論}

\subsection{数式}


\subsection{図・写真}


\begin{center}図1 ○○○○\end{center}

のように,図の下に記す.表のキャプションは

\begin{center}表1 ○○○○\end{center}

のように,表の上に記す.

\section{実験と評価}

\section{考察}

\section{結言}

\begin{thebibliography}{99}
\bibitem{marumaru}
	○○△△, システム情報科学会論文誌, 2, 13-19, 2002.
\bibitem{abc}
	A.B.Cdddddd, J. Systems Information Science, 11, 1145-1159, 2001.
\bibitem{batubatu}
	○○××, □□△△, システム情報科学, ☆☆出版, 1999, 20-21.
\bibitem{efghij}
	E.Fggg and H.Ijjj, Electrical Engineering, KKPress, 2003, 281-284.
\end{thebibliography}
\end{document}
%
%
% EOF 
